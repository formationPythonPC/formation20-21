%\href{https://github.com/formationPythonPC/formation20-21/tree/master/activites-2nde/tp-astrophysicien/donnees-planetes.csv}{\underline{\texttt{donnees-planetes.csv}}}
\documentclass[11pt]{article}
\usepackage{fourier}
\usepackage{hyperref}
\usepackage{mon_paquet}
\usepackage{mon_ds}
%graphes, physique-chimie
%\usepackage{pst-osci}\usepackage{pst-diffraction}\usepackage{pst-circ}\usepackage{pst-dosage}\usepackage{pst-labo}\usepackage{pst-optic}\usepackage{pst-spectra,pstricks-add}\usepackage[pictex]{m-ch-en}\usepackage{m-pictex,m-ch-en}
%\usepackage{psfrag}\usepackage{color,colortbl}\usepackage[table]{xcolor}\usepackage{graphicx}\usepackage[usenames,dvipsnames]{pstricks}\usepackage{epsfig}\usepackage{pst-grad}\usepackage{pst-plot}

%mise en page, multicolonnage, paysage, tableaux maths
%\usepackage{array}\usepackage{multirow}\usepackage{lscape}\usepackage{multido}

%paquets divers, maths, encadrement, symboles, cursif, qcm,lettrine,ombrage
%\usepackage{lettrine}\usepackage{eurosym}\usepackage{amsmath,amssymb,mathrsfs}\usepackage{esvect}\usepackage{esdiff}\usepackage{cancel}\usepackage{fancybox}\usepackage{shadow}\usepackage{pifont}\usepackage{fourier-orns}\usepackage{frcursive}\usepackage{bm}\usepackage{alterqcm}\usepackage{pstricks,pst-3d}(pour les ombres)\usepackage{enumitem}
%\renewcommand\thesection{\Roman{section}}
\usepackage{pgfplots}
\pgfplotsset{width=10cm,compat=1.9}

% POUR LE SIGNE COUP DE POUCE
\usepackage{fontawesome}
\usepackage{fancybox}

\title{Record du monde du 100 mètres}\author{}\date{}
\begin{document}
\maketitle
\thispagestyle{fancy}

\vspace{-1cm}
Le 16 août 2009, un nouveau record du monde du 100 mètres est établi par Usain Bolt en finale des championnats du monde, à Berlin en Allemagne.

Dans le sujet, on teste une simulation du mouvement d'Usain Bolt.



\subsection*{1 -- Vitesse moyenne d'Usain Bolt}

\begin{enumerate}
	\item Visionnez \href{https://github.com/formationPythonPC/formation20-21/tree/master/activites-2nde/2-tp-bolt/Bolt2.avi}{\underline{\texttt{la vidéo de l'épreuve}}}.
 
 \smallskip
 \item \ar Quel référentiel a été choisi dans la vidéo pour suivre le mouvement ?\ajar{1}
 
 \medskip
 
 \item \rea Calculez la vitesse moyenne d'Usain Bolt sur la course, en m.s$^{-1}$ et en km.h$^{-1}$. \ajrea{2.5}
\end{enumerate}


\subsection*{2 -- Les différentes vitesses d'Usain Bolt}

\subsubsection*{2.1 -- Qualifier le mouvement}

\begin{enumerate}
 \setcounter{enumi}{3}
 \item \app Que vaut la vitesse initiale d'Usain Bolt au "top départ" ?\ajar{1}
 
 
 \medskip
 \item \rco Parmi les adjectifs suivants, quels sont ceux qui qualifient le mouvement d'Usain Bolt au cours de la course pour un spectateur placé dans les tribunes ? \ajrco{1}
 
 
 \begin{multicols}{6}
 \begin{itemize}
  \item circulaire
  \item rectiligne
  \item parabolique
  \item uniforme
  \item accéléré
  \item ralenti
 \end{itemize}
\end{multicols}
 
 \end{enumerate}
 
 
 \subsubsection*{2.2 -- Simuler le déplacement}
\begin{enumerate}
 \setcounter{enumi}{5}
 
%\href{https://github.com/formationPythonPC/formation20-21/tree/master/activites-2nde/tp-astrophysicien/simulation1.py}{\underline{\texttt{simulation1.py}}}
 \item Ouvrez alors le fichier \href{https://github.com/formationPythonPC/formation20-21/tree/master/activites-2nde/2-tp-bolt/modelisation1.py}{\underline{\texttt{modelisation1.py}}} avec IDLE ou EduPython. Compilez-le (touche \texttt{F5}).
 
 \smallskip
 \item \ar Quelles sont les informations quant à Usain Bolt qui ont été perdues par la modélisation ? \ajar{1}
 
 \medskip
 \item \app \com Que représentent dans la réalité les listes \tt{T}, \tt{X} et \tt{Y} lignes 9-10-11 ?\ajapp{0.5}\ajcom{0.5}
 
 \begin{center}
  \fbox{\textsc{appel prof.}}
 \end{center}
 
 
  \medskip
  
  \item\app Quelle(s) ligne(s) permet(tent) de \underline{représenter} la trajectoire du point ? \ajapp{0.5}
  
 \medskip
 \item \ar \va À quoi sert la ligne 57 : \texttt{plt.pause(delta\_t)} ? Comment le vérifier ?\ajar{1}\ajva{1}%si on commente, on a juste l'image de la fin / sans commenter, on a toutes les images

 \medskip
\item \va Cette modélisation permet-elle de retrouver les résultats de la question 4. ? Que pensez-vous alors de cette simulation ? \ajva{2}
 
 \medskip
 
 \item \ar \rea Comment calculer la vitesse d'Usain Bolt au 4\textsuperscript{ème} point de la trajectoire (indice 3 pour Python) ? 
 
 Réalisez ce calcul. \ajar{1}\ajrea{1} 
 \cpdepce

 \begin{center}
  \fbox{\textsc{appel prof.}}
 \end{center}
 
 
 \medskip
 
 \item \app \com En utilisant les listes \texttt{X} et \texttt{T} du programme, trouvez et complétez la formule permettant de retrouver ce résultat ligne 63 : \texttt{vx = \ldots}. \hspace{1cm}\danger\ \textsl{Votre formule ne doit contenir aucun chiffre} \danger\hspace{1cm}\cpdepce

 
 
 \ajapp{1}\ajcom{1}
 
 
 % que les listes X et T et les indices de ces listes
  %Python numérote à partir de l'\textit{indice} "0"  : ainsi le 3\textsuperscript{ème} point de la liste \texttt{X} a pour indice 2 et il s'écrit \texttt{X[2]}.
 
 \medskip
 
 \item \rea Faites afficher ce résultat et son unité sur la console en écrivant la commande adéquate ligne 69. \ajrea{0.5}
 
 
 
 
 
\end{enumerate}


\bigskip

\grilleDS
































%\href{https://github.com/formationPythonPC/formation20-21/tree/master/activites-2nde/tp-astrophysicien/simulation1.py}{\underline{\texttt{simulation1.py}}}











\end{document}


\dotfill
\begin{center}
\begin{tabular}{|c|p{2cm}|p{2cm}|p{2cm}|p{2cm}|p{2cm}|}
\hline
compétences&\centering \rco &\centering \rea (schéma)&\centering \rea (calculs)&\centering  \ana &\centering \val \tabularnewline
\hline\hline
points& & & & & \\
obtenus& & & & & \\
\hline
total points&\centering 2,5&\centering 5 &\centering 7,5 &\centering 1&\centering 1\tabularnewline
\hline
 
\end{tabular}             
