%\href{https://github.com/formationPythonPC/formation20-21/tree/master/activites-2nde/tp-astrophysicien/donnees-planetes.csv}{\underline{\texttt{donnees-planetes.csv}}}
\documentclass[12pt]{article}
\usepackage{fourier}
\usepackage{mon_paquet}
\usepackage{hyperref}
%\lhead{}\rhead{}\lfoot{}\rfoot{\LaTeXe{}}
\usepackage{mon_ds}
%\usepackage{ma_programmation}

%graphes, physique-chimie
%\usepackage{pst-osci}\usepackage{pst-diffraction}\usepackage{pst-circ}\usepackage{pst-dosage}\usepackage{pst-labo}\usepackage{pst-optic}\usepackage{pst-spectra,pstricks-add}\usepackage[pictex]{m-ch-en}\usepackage{m-pictex,m-ch-en}
%\usepackage{psfrag}\usepackage{color,colortbl}\usepackage[table]{xcolor}\usepackage{graphicx}\usepackage[usenames,dvipsnames]{pstricks}\usepackage{epsfig}\usepackage{pst-grad}\usepackage{pst-plot}

%mise en page, multicolonnage, paysage, tableaux maths
%\usepackage{array}\usepackage{multirow}\usepackage{lscape}\usepackage{multido}

%paquets divers, maths, encadrement, symboles, cursif, qcm,lettrine,ombrage
%\usepackage{lettrine}\usepackage{eurosym}\usepackage{amsmath,amssymb,mathrsfs}\usepackage{esvect}\usepackage{esdiff}\usepackage{cancel}\usepackage{fancybox}\usepackage{shadow}\usepackage{pifont}\usepackage{fourier-orns}\usepackage{frcursive}\usepackage{bm}\usepackage{alterqcm}\usepackage{pstricks,pst-3d}(pour les ombres)\usepackage{enumitem}
%\renewcommand\thesection{\Roman{section}}


\title{Du sport dans toutes les dimensions}\author{}\date{}
\begin{document}
\maketitle
\thispagestyle{fancy}

\section{D'abord\ldots de l'athlétisme}

\subsection{Le mouvement d'Usain Bolt}
On revient tout d'abord sur le record du monde du 100 mètres d'Usain Bolt.

\begin{enumerate}
\item \rco Revisionnez \href{https://github.com/formationPythonPC/formation20-21/tree/master/activites-2nde/3-tp-sports/Bolt2.avi}{\underline{\texttt{la vidéo}}}; comment peut-on qualifier le mouvement d'Usain Bolt lors de la course ?\ajrco{2}
\end{enumerate}


\subsection{La vitesse d'Usain Bolt}
%\href{https://github.com/formationPythonPC/formation20-21/tree/master/activites-2nde/3-tp-sports/Bolt2.avi}{\underline{\texttt{la vidéo}}}
Ouvrez à présent le fichier  \href{https://github.com/formationPythonPC/formation20-21/tree/master/activites-2nde/3-tp-sports/bolt.py}{\underline{\texttt{bolt.py}}}.
\begin{enumerate}
 \setcounter{enumi}{1}
 
 \item \app Quelle ligne du programme permet de représenter la trajectoire d'Usain Bolt ?\ajapp{1}
 
  \item \ar \rea Rappelez comment calculer la valeur de la vitesse d'Usain Bolt au 6\textsuperscript{ème} point de la trajectoire ? 
 
 Réalisez ce calcul. \ajar{1}\ajrea{1} 
 \cpdepce

 \begin{center}
  \fbox{\textsc{appel prof. 1}}
 \end{center}
 
 
 \medskip
 
 \item \app \com En utilisant les listes \texttt{X} et \texttt{T} du programme, trouvez et complétez la formule permettant de retrouver ce résultat ligne 55 : \texttt{vx = \ldots}. \hspace{1cm}\danger\ \textsl{Votre formule ne doit contenir aucun chiffre} \danger\hspace{1cm}
 
 \cpdepce

 
 
 \ajapp{1}\ajcom{1}
 
 
 % que les listes X et T et les indices de ces listes
  %Python numérote à partir de l'\textit{indice} "0"  : ainsi le 3\textsuperscript{ème} point de la liste \texttt{X} a pour indice 2 et il s'écrit \texttt{X[2]}.
 
 \medskip
 
 \item \rea\com Faites afficher ce résultat et son unité sur la console en écrivant la commande adéquate ligne 63. \ajrea{0.5}\ajcom{0.5}
 

 \item \ar Selon vous, quelle serait l'allure du vecteur vitesse au 6\textsuperscript{ème} point de la trajectoire ?\ajar{1}
 
 
 \item \ar \com \va On souhaite représenter le vecteur vitesse au 6\textsuperscript{ème} point de la trajectoire. Complétez les lignes 68 à 71 sachant que :
 \begin{itemize}
  \item \tt{x} représente l'abscisse du 6\textsuperscript{ème} point de la trajectoire
  \item \tt{y} représente l'abscisse du 6\textsuperscript{ème} point de la trajectoire
  \item \tt{vx} représente la vitesse selon l'axe des abscisses du 6\textsuperscript{ème} point de la trajectoire
  \item \tt{vy} représente la vitesse selon l'axe des ordonnées du 6\textsuperscript{ème} point de la trajectoire
  \end{itemize}
\ajar{1}\ajcom{2}

\cpdepce

$\Longrightarrow$ Décommentez alors toutes les lignes 68 à 71 ainsi que la ligne 74. À quoi sert cette ligne ? Que pensez-vous du résultat obtenu ?\ajva{2}

 \begin{center}
  \fbox{\textsc{appel prof. 2}}
 \end{center}


\end{enumerate}








\section{Et maintenant \ldots du basket}

Visionnez la vidéo \href{https://github.com/formationPythonPC/formation20-21/tree/master/activites-2nde/3-tp-sports/basket.avi}{\underline{\texttt{basket.avi}}}. Une chronophotographie de cette vidéo a été réalisée, on a obtenu les données du fichier \href{https://github.com/formationPythonPC/formation20-21/tree/master/activites-2nde/3-tp-sports/donnees-basket2.csv}{\underline{\texttt{donnees-basket2.csv}}}.

\subsection{Données de la chronophotographie}
\begin{enumerate}
 \setcounter{enumi}{7}
 \item \rco Quel est le référentiel associé à l'étude du mouvement du ballon ?\ajrco{1}
 \item \app Ouvrez le fichier de données ; quelles sont les informations qu'il contient ?\ajapp{1}
 \item \app Que vaut la période d'échantillonnage des mesures de la chronophotographie ?\ajapp{1}
\end{enumerate}

\subsection{Un lancer franc}
\begin{enumerate}
 \setcounter{enumi}{10}
 \item \app \com \va Ouvrez à présent le fichier \href{https://github.com/formationPythonPC/formation20-21/tree/master/activites-2nde/3-tp-sports/trajectoire.py}{\underline{\texttt{trajectoire.py}}}. Entrez à la ligne 10 la commande permettant de représenter les points de la chronophotographie sur le graphe. Comment vérifiez votre réponse ? \ajapp{0.5}\ajcom{1}\ajva{0.5}
 
 \item \ar \com En vous servant de ce qui a été fait précédemment, tracez sur le graphe le vecteur vitesse au premier point de la trajectoire. Votre commande sera rentrée ligne 27.

 \textsc{note :} lors de l'utilisation de la fonction \tt{quiver}, on choisira un facteur d'échelle de 18 : \tt{scale = 18}
 
 \cpdepce
 
 \ajar{1}\ajcom{1}
 
 
  \begin{center}
  \fbox{\textsc{appel prof. 3}}
 \end{center}

 
 
 
\item \ar \com \va Les lignes 30 et 31 permettront de tracer tous les vecteurs vitesse pour tous les points où cela est possible ; la variable \tt{i} corespondant à l'indice du point où on trace le vecteur.

Complétez le programme ligne 31 pour obtenir le tracé de tous les vecteurs vitesse des points de la trajectoire.


  \begin{center}
  \fbox{\textsc{appel prof. 4}}
 \end{center}


$\Longrightarrow$ le résultat obtenu vous semble-t-il cohérent ?
 
 \end{enumerate}


\grilleDS


\end{document}

% la grille de fin si DS
\grilleDS





